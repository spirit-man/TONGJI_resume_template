% 从NPU CV修改来的同济版本,感谢https://www.overleaf.com/latex/templates/npu-cv/mncqzxhvfzrx
% 可以自己控制/vspace不同节之间的间距
\iffalse
	前言:
		参考自一个祖传的模板,以及https://github.com/LeyuDame/BNUCV/tree/main 上的BNU的latex简历模板中的代码。
		注意要用XeLaTeX编译链进行编译,且要进行三次编译才能显示照片。问就是LaTeX的锅。
		vscode+latex的话,配置的json中"latex-workshop.latex.recipes"添加:
		{
            "name": "XeLaTeX*3",
            "tools": [
                "xelatex",
                "xelatex",
                "xelatex"
            ]
        }
		即可使用三次XeLaTeX编译。

		LaTeX+VScode怎么配置看https://www.zhihu.com/column/p/166523064。

        每个章节的格式都能混着用,顺序都可以变,只是给了个例子。
        比如你找工作,可以把技能那部分往前挪。
        比如你竞赛经历很多,你就往前挪。
        比如你觉得“其他”有点多余,就删了。

        主要贡献还是把原本word的那个模板页眉页脚和背景完美加进来了。
        LaTeX排版就是很整齐,强迫症狂喜。


        记得要三次XeLaTeX编译!!!
        记得要三次XeLaTeX编译!!!
        记得要三次XeLaTeX编译!!!
        这个很重要,所以说三遍!

        你可以试试两次的,好像某些环境两次编译也能显示图像,但最好还是三次编译。
\fi

\documentclass[4pt]{article}
\usepackage{xltxtra}
\usepackage{bookmark}
\usepackage{hyperref}
\hypersetup{hidelinks}
\usepackage{url}
\urlstyle{tt}
\usepackage{multicol}
\usepackage{xcolor}
\usepackage{calc}
\usepackage{graphicx}
\usepackage{tikz}
\usetikzlibrary{calc}
\usepackage{fontspec}
\usepackage{xeCJK}
\usepackage{relsize}
\usepackage{xspace}
\usepackage{fontawesome}
\usepackage{titlesec}
\usepackage{enumitem}
\usepackage{siunitx}
\usepackage{amssymb}
\usepackage{tabularx}
\usepackage{multicol}
\usepackage{fontspec}
\usepackage{array}
\usepackage{footmisc}

\newcommand{\blfootnote}[1]{%
  \begingroup
  \renewcommand\thefootnote{}\footnote{#1}%
  \addtocounter{footnote}{-1}%
  \endgroup
}
\newcommand{\graytext}[1]{{\color{gray}#1}}
\newcommand{\fsize}[1]{{\csname #1\endcsname}}

% 一些小设置,参考自https://github.com/LeyuDame/BNUCV/tree/main
\CJKsetecglue{}							            % 取消中文字符与数字之间的间隔
\protected\def\Cpp{{C\nolinebreak[4]\hspace{-.05em}\raisebox{.28ex}{\relsize{-1}++}}\xspace}	% 这是个更好看的C++写法,你直接写C++的话,+号会很大,可以使用\Cpp来代替
\setlength{\parindent}{0pt}							% 取消全局段落缩进
\pagenumbering{gobble}								% 取消页码显示
%\setlist{noitemsep}									% 禁用列表中项目之间的额外垂直间距,但保留列表周围的间距
%\setlist{nosep}										% 禁用列表中项目之间的额外垂直间距及列表周围的间距
% \setlist[itemize]{topsep=0em, leftmargin=*}		% 增加了itemize顶部间距
% \setlist[enumerate]{topsep=-2em, leftmargin=*}	% 增加了enumerate顶部间距
\setlist[itemize]{topsep=0em, itemsep=0em, parsep=0em, leftmargin=*}		% itemsep控制项目间距
\setlist[enumerate]{topsep=0em, itemsep=0em, parsep=0em, leftmargin=*}	

\titleformat{\section}					    % 将原标题前面的数字取消了
  {\large\bfseries\raggedright} 		      % 字体改为LARGE,bold,左对齐
  {}{0em}                      			  % 可用于添加全局标题前缀
  {}                           			  % 可用于添加代码
  [{\color{NPU_Blue}\titlerule}]            % 标题下方加一条线
\titlespacing*{\section}{0cm}{*1.2}{*1.2}	% 标题左边留白,上方1.2倍,下方1.2倍

\titleformat{\subsection}				    % 将原二级标题前面的数字取消了
  {\large\bfseries\raggedright} 		      % 字体改为large,bold,左对齐
  {}{0em}                      			  % 可用于添加全局二级标题前缀
  {}                           			  % 可用于添加代码
  []
\titlespacing*{\subsection}{0cm}{*0.8}{*0.2}% 二级标题左边留白,上方1.2倍,下方1.2倍

% 页面大小与页边距,按需求调整
\usepackage[
	a4paper,
	left=1.2cm,
	right=1.2cm,
	top=1.5cm,
	bottom=1cm,
	nohead
]{geometry}

% 中文字符间距
\renewcommand{\CJKglue}{\hskip 0.05em}

% % 英文字体
% \setmainfont[
%     Path=fonts/,
%     Extension=.ttf,
%     BoldFont=* Bold,
% ]{Microsoft Yahei}
% % 中文字体
% \setCJKmainfont[
%     Path=fonts/,
%     Extension=.ttf,
%     BoldFont=* Bold,
% ]{Microsoft Yahei}

% % 英文字体
% \setmainfont[
%     Path=fonts/,
%     Extension=.otf,
%     AutoFakeBold=true,
% ]{Times-New-Roman}
\setmainfont[
    Path=fonts/,
    Extension=.otf,
    AutoFakeBold=1.5,  % 增加数值使字体更粗
    BoldFeatures={FakeBold=2},  % 专门控制粗体的粗度
]{Times-New-Roman}

% 中文字体
\setCJKmainfont[
    Path=fonts/,
    Extension=.otf,
    AutoFakeBold=true,
]{AdobeSongStd-Light}


% 主题色
% 西工大蓝
% \definecolor{NPU_Blue}{RGB}{0, 80, 158}
\definecolor{NPU_Blue}{RGB}{0, 96, 151}
% 这里把表格的行间距调成1.2倍了
\renewcommand{\arraystretch}{1.2}
% 这里把正文的行间距调成1.2倍了
\linespread{1.2}

%%%%%%%%%%%%%%%%%%%%%%%%%%%%%%%%%%%%%%%%%%%%%%%%%%%%%%%%%%%%%%%%%%%%%%%%%%%%%%%%%%%%%%%%%%%%%%%%%%%%%%%%%%%%%%%%%%%%%%%%%%%%%%%%%%%%%%%%%%%%%%%%%%
%    !!!!!!!! 记得改这里 !!!!!!!!
%%%%%%%%%%%%%%%%%%%%%%%%%%%%%%%%%%%%%%%%%%%%%%%%%%%%%%%%%%%%%%%%%%%%%%%%%%%%%%%%%%%%%%%%%%%%%%%%%%%%%%%%%%%%%%%%%%%%%%%%%%%%%%%%%%%%%%%%%%%%%%%%%%
% 学院
\newcommand{\school}{\small{下北泽豪俊影视学院 | Shimokitazawa Legendary Film Academy (SLFA)}}
% 也可以不写英语
%\newcommand{\school}{电子信息学院}
% 联系方式
\newcommand{\contact}
{
    \small              % 换了更小的字号
    % \footnotesize       % 这比上面的小一号
    \scriptsize         % 这比上面的再小一号
    \textcolor{white}
    {
        \faEnvelope \quad \href{mailto:114514@tongji.edu.cn}{114514@tongji.edu.cn}    % 邮箱,前面的超链接可以直达邮箱软件
        \hspace{4em}    % 这里可以调间距
        \faWechat \quad senpeiiski               % 微信
        \hspace{4em}    % 这里可以调间距
        \faPhone \quad 1919810                  % 手机号
        \hspace{4em}    % 这里可以调间距
        \faGithub \quad \href{https://github.com/spirit-man}{https://github.com/spirit-man}         % github
    }
}



%%%%%%%%%%%%%%%%%%%%%%%%%%%%%%%%%%%%%%%%%%%%%%%%%%%%%%%%%%%%%%%%%%%%%%%%%%%%%%%%%%%%%%%%%%%%%%%%%%%%%%%%%%%%%%%%%%%%%%%%%%%%%%%%%%%%%%%%%%%%%%%%%%
%    !!!!!!!! 这里开始就是正文了 !!!!!!!!
%%%%%%%%%%%%%%%%%%%%%%%%%%%%%%%%%%%%%%%%%%%%%%%%%%%%%%%%%%%%%%%%%%%%%%%%%%%%%%%%%%%%%%%%%%%%%%%%%%%%%%%%%%%%%%%%%%%%%%%%%%%%%%%%%%%%%%%%%%%%%%%%%%
\begin{document}
	% 如果有多页简历,请把页眉页脚和背景复制粘贴到第二页的内容之前
	% 页眉,校徽,学院名
	\begin{tikzpicture}[remember picture, overlay]
		\node[anchor=north, inner sep=0pt](header) at (current page.north){
			\includegraphics[width=\paperwidth]{images/header.png}
		};
		\node[anchor=west](school_logo) at (header.west){
			\hspace{0.5cm}
			\includegraphics[width=0.2\textwidth]{images/logowhite.png}
		};
		\node[anchor=east](school_name) at(header.east){
			\textcolor{white}{\textbf{\school}}
			\hspace{0.5cm}
		};
	\end{tikzpicture}
	\vspace{-4em}

	% 页脚,联系方式
	\begin{tikzpicture}[remember picture, overlay]
		\node[anchor=south, inner sep=0pt](footer) at (current page.south){
			\includegraphics[width=\paperwidth]{images/footer.png}
		};
        % 联系方式
        \node[anchor=center] at(footer.center){\contact};
	\end{tikzpicture}
	
	% 背景
	\begin{tikzpicture}[remember picture, overlay]
		\node[opacity=0] at(current page.center){
			\includegraphics[width=0.7\paperwidth, keepaspectratio]{images/badge.png}
		};
	\end{tikzpicture}
    

    \begin{figure}[h]
    % 左半边,信息,比例可以调整
    \begin{minipage}{0.87\textwidth}
        \vspace{-1.8em}
        % \section{\makebox[\widthof{\faGraduationCap}][c]{\color{NPU_Blue}{\faUser}}\quad 个人信息}
        \begin{center}
            % 名字居中显示,字号放大
            {\huge\bfseries 李田所}
            
            \vspace{0.5em} % 调整名字和下方信息的间距
            1919810\quad$\vert$ \quad \href{mailto:114514@tongji.edu.cn} {114514@tongji.edu.cn}\quad$\vert$ \quad 上海市\\
            研二 \quad$\vert$ \quad 出生日期:2001年5月14日 \quad$\vert$ \quad 求职意向:国家特级演员/量化调酒师
            
        \end{center}
    \end{minipage}
    % 右半边,照片
    \begin{minipage}{0.12\textwidth}
        \vspace{-0.1em}
        \includegraphics[width=\linewidth]{images/先辈帅气版.png}
    \end{minipage}
    \end{figure}


    \vspace{-3em}
	% 教育背景
    	% \faGraduationCap这类\fa开头的都是font awesome里的logo,想换成其他logo的话,可以看一下附带的fontawsome.pdf,自行替换。
		% \section{\makebox[\widthof{     这里!    }][c]{\color{NPU_Blue}{     和这里!    }}\quad 标题}
	\section{\makebox[\widthof{\faGraduationCap}][c]{\color{NPU_Blue}{\faGraduationCap}}\quad 教育背景}
	\vspace{-1.5em}
    
    \begin{table}[h!]
    
        \hspace{-0.9em}
        \begin{tabularx}{\textwidth}{p{\widthof{同济大学下北泽豪俊影视学院}}p{\widthof{行为艺术硕士}}p{\widthof{2019年9月-2023年7月}}X}
        
        \textbf{同济大学下北泽豪俊影视学院}     & \textbf{表演学士}        & 2019年9月-2023年7月     & GPA: 10/9 \textbf{(排名: 114/514)}    \\
        \multicolumn{4}{l}{课程: 红茶品鉴与应急处理基础,野兽咆哮发声技巧,毕业设计:《速达优》,小学期:校园突击演出等 }                           \vspace{0.5em} \\ 
        \textbf{同济大学下北泽豪俊影视学院}     & \textbf{行为艺术硕士}     & 2023年9月-2026年7月     & GPA: 9/10                \\
        \multicolumn{4}{l}{课程: 先锋行为艺术理论,迷因符号学研究方法,行为艺术田野调查方法,极限空间行为艺术等}
        \end{tabularx}

    \end{table}

    \vspace{-1em}
    % 项目经历(找导师一般都看中这个),可以改成“科研经历”
        % \faGears 这是齿轮,适合机械类,我电信的也喜欢齿轮,就用这个了
        % \faFlask 这是烧瓶,适合生化类
        % \faLaptop 这是个笔记本电脑,适合计算机类
        % \faUsers 这是三个人,适合商科
    \section{\makebox[\widthof{\faGraduationCap}][c]{\color{NPU_Blue}{\faLaptop}}\quad 项目经历}
    % \vspace{0.5em}
    % 小技巧,老师想看的重点加粗,比如商科类的一般更想看到数字,工科类的更想看到技术
    \subsection{Research on Wild Beast Aesthetics and Its Impact on Contemporary Performance Art \hfill SCI期刊-在审}
    \textbf{第一作者,独立完成全部研究工作} \hfill 2024年1月至今

    \begin{itemize}
        \item 基于\textbf{野兽美学理论与表演艺术学},构建了包含突发性、魔性度、浮夸度三个维度的\textbf{野兽表演评价体系}。{\textsuperscript{\textbf{\color{red}*}}}
        \item 提出\textbf{多模态野兽表演特征提取方法},在全球114个表演数据集上进行实验,表演魔性度评分较传统方法提升514\%。
    \end{itemize}

    
    % 这是个工科类加粗的例子
    \vspace{0.5em}                % 这是换行用的
    \subsection{突发事件智能应对与红茶应急处理系统 \hfill 国家发明专利}

    \textbf{独立开发,项目负责人} \hfill 2023年10月至2024年3月

    \begin{itemize}
        \item 设计了基于\textbf{深度学习的红茶品质评估算法},实现了对红茶温度、浓度的实时监控,准确率达到114.514\%。
        \item 开发了\textbf{多场景突发事件处理模块},支持包括\textbf{速速进行、一般巡查、紧急咆哮}在内的24种应急预案。
    \end{itemize}


    % 不知道写啥好
    \vspace{0.5em}                % 这是换行用的
    \subsection{野兽咆哮特征识别与分析系统 \hfill 国家级大学生创新创业项目}
        
    \textbf{项目组长,带领3人团队} \hfill 2023年7月至2023年12月

    % 在研的也能写
    % 课程大作业也能写,但是不要标明是大作业就行
    \begin{itemize}
        \item 构建了\textbf{野兽咆哮音频数据库},包含超过1919个高质量样本,覆盖810种不同场景下的咆哮特征。
        \item 实现了基于\textbf{深度学习的咆哮特征提取与分类系统},支持实时识别与评分,在校园突发事件演出中应用效果显著。
    \end{itemize}
    
    
    % 竞赛经历(找导师也可能看中这个,因为代表了一定实践能力,但是尽量对口吧,不要运动会都写进去了)
    \section{\makebox[\widthof{\faTrophy}][c]{\color{NPU_Blue}{\faTrophy}}\quad 竞赛经历}
    \vspace{-1em}
    \begin{table}[h!]
        \begin{tabularx}{\textwidth}{Xp{\widthof{首席王爷}}p{\widthof{皇室认证-特等奖}}p{\widthof{2024年5月}}}
            \textbf{2024年全国王爷杯贵族风度大赛} & 首席王爷 & 皇室认证-特等奖 & 2024年5月 \\
            \textbf{2023年迫真空手道锦标赛} & 真空担当 & 宇宙级-金牌 & 2023年11月\\
            \textbf{2024年国际提肛艺术表演大赛} & 个人参赛 & 全球级-卫冕冠军 & 2024年2月\\
        \end{tabularx}
    \end{table}
    
    \vspace{-1em}
    % 其他(也是看你想不想写)
    \section{\makebox[\widthof{\faGraduationCap}][c]{\color{NPU_Blue}{\faStar}}\quad 所获荣誉}
    \newlength{\datewidth}
    \settowidth{\datewidth}{2024年10月}
    \newcommand{\dateitem}[2]{\item \makebox[\datewidth][l]{#1}\hspace{2.5em}#2}
    \begin{itemize}
        \dateitem{2024年10月}{获得\textbf{迫真空手道部年度最佳真空雷普奖}}
        \dateitem{2024年7月}{荣获\textbf{会员制餐厅114514号终身VIP会员认证}}
        \dateitem{2024年5月}{担任\textbf{下北泽野兽足球队主力前锋},赛季进球1919个}
        \dateitem{2024年3月}{获评\textbf{下北泽第二小学优秀课后辅导员},深受小朋友喜爱}
        \dateitem{2023年12月}{接受\textbf{《野兽风云》杂志专访},并被评为年度最具影响力人物}
    \end{itemize}
    
    % \vspace{-1em}
    % 其他(也是看你想不想写)
    
    % 技能特长,上面写很多的话,这里就随便写点,反正上面都看出来了。上面写的不多的话,这里着重强调你会什么。
    % 哦,你找工作的话,这里多写点,记得对口,可以\textbf{}加粗。
    % 这里能吹牛皮就吹牛皮,但是确保面试的时候别露馅就行。
    \section{\makebox[\widthof{\faGraduationCap}][c]{\color{NPU_Blue}{\faWrench}}\quad 技能特长}
    % \vspace{0.5em}
    \begin{itemize}
        \item 编程能力:熟练使用{\Cpp} 、Python、MATLAB、Shell编程语言与 {\LaTeX}排版。
        \item 英语水平:雅思2.4,四级114,六级514。
        \item 专业能力:精通\textbf{迫真点云调教,野兽操作系统,24帧优化大师,先辈级HOMO姿态估计,深度恶臭学习框架等},熟练掌握\textbf{哲学级调试工具,沼气泄露探测器等}。
        \item 技术视野:深入研究多器官联合感知技术、深度恶臭学习算法,精通\textbf{淫梦语言模型、视觉哲学模型、114514维度具身智能}等前沿技术。
    \end{itemize}

    % % 所获荣誉(这个看你想不想写了)
    % \section{\makebox[\widthof{\faStar}][c]{\color{NPU_Blue}{\faStar}}\quad 所获荣誉}
    % \vspace{-1em}
    % \begin{multicols}{2}
    %     \begin{itemize}
    %         \item 某年学业先进个人
    %         \item 某年某奖学金某等奖
    %         \item 某大使
    %         \item 某年某奖学金某等奖
    %         \item 某年优秀团员称号
    %         \item 某年某称号
    %     \end{itemize}
    % \end{multicols}

    
\vfill
\vspace{1em}
\noindent{\color{NPU_Blue}\rule{\textwidth}{0.4pt}}
\vspace{0.3em}
\noindent\footnotesize{$^1$标注{\textsuperscript{\textbf{\color{red}*}}}
的项目已开源至本人GitHub账号:https://github.com/spirit-man},欢迎批评指教!
\end{document}
